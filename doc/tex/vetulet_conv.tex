
\title{ EOV - HD72 - WGS84}
\author{
	Zoltan Boncz - Binh Nguyen\\
	\\
	\underline{ELTE-Department of Computer Science}
}
\date{\today}

\documentclass[10pt]{article}
\usepackage[utf8]{inputenc} 
\usepackage[T1]{fontenc}
\usepackage{mathtools} 


\begin{document}
\newpage
\maketitle

\begin{abstract}
	Intended for developers, who are using EOV-WGS84 conversion library. 
	Some of the major steps and constants are defined in nutshell.
\end{abstract}

\newpage

\section{Introduction}
	Before getting to coding section, we would like to mention a few words on projection systems. Namely EOV, HD72 and WGS84.
	
\paragraph{EOV - Egységes Országos Vetületi rendszer}
	A projection system used uniformly for Hungarian civilian base maps, in general, for spatial informatics.
	Cylinder projection in traversal position. Established in 1972-1975.
	
\paragraph{HD72 - Hungarian Datum of 1972}
	Hungarian Geodetic Reference System defines procedures for adjustment of the Hungarian Astrogeodetic Network, 
	and on the IUGG GRS 1967 reference ellipsoid, as written for the deflections of vertical at the Laplace-points.
\paragraph{WGS84 -World Geodetic System}
	The World Geodetic System is a standard for use in cartography, geodesy, and navigation. 
	It comprises a standard coordinate frame for the Earth, a standard spheroidal reference surface 
	(the datum or reference ellipsoid) for raw altitude data, and a gravitational equipotential 
	surface (the geoid) that defines the nominal sea level. It's latest revision is WGS84 last revised in 2004, which 
	will be valid till 2010. Some of it's predecessors are:
	\begin{itemize}
		\item WGS 72
		\item WGS 66
		\item WGS 60
	\end{itemize}

\section{Environment}\label{previous work}
	Conversion library is written in Ruby programming language. Math library is required for mathematical calculations.
	Before using this library user should check if they have ruby and math lib on their running environment.
	
\section{EOV - Contants}\label{eov}
	In this section we describe the results.
	\subsection{Reference surface}
	Let \\
	$$ a = 6 378 160.000 m $$
	$$ b = 6 356 774.516 m $$
	\\
	Ellipsodial Geographic Coordinates are: $\Phi$ and $\Lambda$
	
	\subsection{Projecting sphere}
	A Gauss-sphere fitted to reference ellipsoid at the normal parallel.
	\\
	It is used for Gauss-sphere:
	It's radius $R = 6 379 743.001 m$ and and its spherical geographic coordinates:
	\\
	$\phi$ and $\lambda$
	
	\subsection{Normal parallel}
	Defined as average of latitudes for Hungary, with ellipsoidal and spherical latitudes below
	\\
	$\Phi_n = 47.10'00.0000"$
	$\phi_n = 47.07'20.0578"$
	
	\subsection{Origin of geographic coordinates}
	The Equator and the Greenwich meridian
	
	\subsection{Origin of projection}
	A point on the meridian of site Gellérthegy:
	\\
	$\phi_0 = 47.06'00"$
	$\lambda_0 = 0.0'0"$
	\\
	The plane coordinates are:
	\begin{itemize}
		\item $x_0 = 0.0 m$
		\item $y_0 = 0.0 m$
	\end{itemize}
	and by computing the ellipsoidal geographic coordinates:
	\\
	$\Phi_0 = 47.08'39.8174"$
	\\
	$\Lambda_0 = 19.02'54.8584"$

	\subsection{Displacement}
	Origin of the EOV plane coordinates is moved to South-West with respect to origin of projection 
	such that all possible EOV coordinates be positive for Hungary:
	\\
	$X_{EOV} = x + 200 000.000 m$ \\
	$Y_{EOV} = y + x + 200 000.000 m$ \\
	where x and y are the original coordinates in EOV projection system.
	
	\subsection{Scale factor}
	Factor $m_0 = 0.99993$
	
	\subsection{Linear Distortion}
	Along the y axis: $-7cm/km$ \\
	North of Hungary: $+26cm/km$ \\
	South of Hungary: $+23cm/km$
	
	\subsection{Area Distortion}
	Along the y axis: $-1.4 m2/ha$ \\
	North of Hungary: $+4.5 m2/ha$ \\
	South of Hungary: $+4.5 m2/ha$
	
	
	
\section{HD72 - Contants}\label{hd72}
	Our base ellipsoid is: \emph{GRS67 (IUHH 1967)}
	\subsection{The Molodensky-Badekas type transformation parameters}
	Direction of the transformation is HD72 $\gg$ WGS84 \\
	\begin{center}
		$d_x = +57.01$\\
		$d_y = -69.97$\\
		$d_z = -9.29$\\		
	\end{center}
	
	\subsection{Errors for Molodensky-Badekas type of transformation}
	\begin{itemize}
		\item On average horizontal accuracy: $0.40$ meters.
		\item Maximum horizontal error: $1.00$ meter.
		\item Vertial accuracy: being optimized.
	\end{itemize}
	
	\subsection{For GPS receivers}
	In the User Datum function, the da and df parameters can be computed from the shape difference between
	the GRS67 and the WGS84 ellipsoids and the shift parameters should be rounded to integers as follows:
	\begin{itemize}
		\item $d_x = +57$ meters.
		\item $d_y = -70$ meters.
		\item $d_z = -9$ meters.
		\item $d_a = -23$ meters.
		\item $df = -10^{-8}$ meters.
	\end{itemize}
	
	\subsection{The Bursa-Wolf type transformation parameters}
	Direction of the transformation: HD72 $\gg$ WGS84 \\
	\begin{itemize}
		\item $d_x = +52.684$ meters.
		\item $d_y = -71.194$ meters.
		\item $d_z = -13.975$ meters.
		\item $e_x = +0.3120$ arc seconds.
		\item $e_y = +0.1063$ arc seconds.
		\item $e_z = +0.3729$ arc seconds.
		\item $k = +1.0191$ ppm
		\item Rotation convention: Coordinate frame rotation (most GIS packages use this one).
	\end{itemize}
	
	\subsection{Error for Bursa-Wolf type of transformation}
	Average accuracies:
	\begin{itemize}
		\item Average horizontal accuracy: 0.19 meters.
		\item Maximum horizontal error: 0.41 meters.
		\item Vertical accuracy: optimized.
	\end{itemize}
	\emph{Notice:} Several 7-parameter sets can be defined with almost this accuracy, but take care of
	the signs of the parameters and the rotation convention.
	
	\subsection{Bursa-Wolf Type}
	In our case, Bursa-Wolf Type transformation have been implemented.

\section{Usage - I/O}
	Usage can be found in rdoc html documentation generated by rdoc.
	
	
\section{References}
	Thanks to TIMÁR GÁBOR (timar@ludens.elte.hu) associate professor, head of Geophysics and Space Science Department,
	without his help and reference materials and sample application "bajnok.xls" aka "champion excel sheet",
	we could not accomplish the conversion module.
	
	\emph{Homepage: http://sas2.elte.hu/tg/}

\bibliographystyle{abbrv}
\bibliography{main}

\end{document}